\documentclass{article}
\usepackage[ngerman]{babel}
\usepackage{amsmath, amssymb, amsthm, mathtools}

% === extra Macro ===
\newcommand{\ggT}{\operatorname{ggT}}

% === Kajetan's Macros ===
\newcommand{\fa}{\qquad \forall \quad}
\newcommand{\eps}{\varepsilon}
\newcommand{\im}{{\rm Im}\,}
\newcommand{\re}{{\rm Re}\,}
\newcommand{\grad}{{\mathrm{grad}}}
\newcommand{\esubset}{{\subset \atop \not=}}

\newcommand{\be}{\begin{equation}}
\newcommand{\ee}{\end{equation}}
\newcommand{\bea}{\begin{eqnarray}}
\newcommand{\eea}{\end{eqnarray}}
\newcommand{\beao}{\begin{eqnarray*}}
\newcommand{\eeao}{\end{eqnarray*}}
\newcommand{\nn}{\nonumber}
\newcommand{\ds}{\displaystyle}
\newcommand{\ol}{\overline}
\newcommand{\ul}{\underline}
\newcommand{\fn}{\mathbf{0}}

\newcommand{\K}{\mathbb{K}}
\newcommand{\C}{\mathbb{C}}
\newcommand{\N}{\mathbb{N}}
\newcommand{\Z}{\mathbb{Z}}
\newcommand{\Q}{\mathbb{Q}}
\newcommand{\R}{\mathbb{R}}
\newcommand{\bP}{\mathbb{P}}

\newcommand{\rkl}{\rangle}
\newcommand{\lkl}{\langle}
\newcommand{\etm}{\subsetneq}

\begin{document}
Opa Wichtel hat noch eine Amateurfunk Station und schnappt einen verschlüsselten Funkspruch auf. Die Übertragung könnte interessant sein. Opa Wichtel hat die Zahlen $212919$ notiert.

Ihr vermutet, dass die Nachricht an den Grinch addressiert ist. Sein öffentlicher Schlüssel ist bekannt und lautet $(e, N) := (5, 3 548 669)$.

Tipps:
\begin{itemize}
    \item Grinch hat den Verschlüsselungsexponenten 5 (und nicht 3) gewählt, weil 5 seine Lieblingszahl ist. (Und nicht etwa, weil Aaron erst um 2 in der Nacht gemerkt hat dass $3$ und $\varphi(N)$ nicht teilerfremd sind)
    \item Wolfram Alpha faktorisiert diese Zahl immer noch instant
    \item Wenn du andere Tipps geben willst: Es wird vermutet das Grinch ein Fan der Amerikanischen Unabhängigkeit ist. Daraus könnte man folgern: eine Primzahl ist 1777, es ist nämlich die nächste Primzahl zu 1776.
    \item Die andere Primzahl ist 1997. Damit kann man vielleicht auch was tipsen.
    \item Es gibt Online-Rechner, um modulare inverse zu berechnen.
\end{itemize}
Lösung:
\begin{itemize}
    \item $\phi(N) = (p-1)(q-1) = 3544896$
    \item Modulares Inverses von $5$ mod $\varphi(N)$ ist $2835917$
    \item $212919^{2835917} \equiv 3224624 \pmod N$ 
    \item \texttt{212919**2835917 \% 3548669} dauert schon ein bisschen in Python hihi
    \item In binär $01100010 0110100 00110000_2$
    \item oder die Buchstaben 1, 4, 0
\end{itemize}
\end{document}

